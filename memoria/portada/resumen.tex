\cleardoublepage

\chapter*{Resumen\markboth{Resumen}{Resumen}}

En la actualidad, la ciencia ha avanzado a pasos agigantados con respecto a las soluciones tecnológicas. Especialmente la robótica, también gracias a que abarca una inmensa variedad de campos donde se pueden desarrollar soluciones eficientes y robustas.\\

Además, ha surgido un nuevo paradigma con el uso de drones, o sistemas aéreos provistos de sensores y actuadores, que amplian el espectro de uso para herramientas tecnológicas, permitiendo abordar los problemas desde nuevas perspectivas. En este proyecto, el foco de estudio se centra en los \ac{UAV}, ya que se busca automatizar todo el proceso de manejo del mismo.\\

De este modo, surge la idea de realizar este \ac{TFG}, juntando lo mejor de ambos mundos, soluciones autónomas con dispositivos aéreos tremendamente adaptables a las circunstancias del problema.\\

Concretamente, el objetivo de este proyecto ha sido robotizar un dron, con el fin de rastrear y navegar una señal de \ac{RF}, mediante el uso de distintos algoritmos, dentro de la infraestructura proporcionada por \ac{ROS}, con el fin de realizar una comparativa y determinar cual desempeña mejor su papel.