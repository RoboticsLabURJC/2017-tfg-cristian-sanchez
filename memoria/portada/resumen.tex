\cleardoublepage

\chapter*{Resumen\markboth{Resumen}{Resumen}}

En la actualidad, la ciencia ha avanzado a pasos agigantados con respecto a las soluciones tecnológicas. Especialmente la robótica y la \ac{IA}, gracias también a que abarca una inmensa variedad de campos donde se pueden desarrollar soluciones eficientes y robustas. Uno de los puntos más destacables es la generación de comportamientos autónomos, lo que dota a estos sistemas de una independencia muy favorable a la hora de resolver tareas complejas, vease por ejemplo en labores de patrullaje, para controlar ciertas zonas, o también en agricultura, donde se requiera controlar que zonas deben regarse más y cuales menos, entre otros. \\

Adicionalmente, nos encontramos el uso de drones, o sistemas aéreos provistos de sensores y actuadores, que amplían el abanico de uso en el entorno tecnológico, permitiendo abordar los problemas desde nuevas perspectivas. En este proyecto, el foco de estudio se centra en los \ac{UAV}, cuyo uso abarca desde rescates en áreas poco accesibles, hasta inspecciones en la industria, o labores de inventariado, entre muchos otros ejemplos.\\

En paralelo, tenemos el estudio del espectro de \ac{RF}, el cual abarca gran parte del mundo moderno, como puede ser en el uso de comunicaciones móviles, o en la transmisión usando bandas de frecuencia como la radio FM. Por ello, comprender y trabajar sobre la propagación de señales puede introducirnos en nuevas aplicaciones útiles para el mundo.\\

De este modo, surge la idea de realizar este \ac{TFG}, agrupando cada parte descrita anteriormente, es decir, soluciones autónomas con dispositivos aéreos tremendamente adaptables a las circunstancias de un problema derivado de señales \ac{RF}. Concretamente, el objetivo de este proyecto es demostrar que, empleando aprendizaje por refuerzo (Q-Learning), se puede lograr rastrear y navegar hacia una el transmisor de una señal \ac{RF}, utilizando un dron situado en un escenario realista (con obstáculos y degradación de señal), de forma efectiva y robusta.