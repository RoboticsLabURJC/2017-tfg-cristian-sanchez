\chapter{Anexo}
\label{cap:anexo}

A continuación muestro una serie de conceptos básicos relacionados con el estudio del procesamiento de señales\footnote[1]{Información extraida de la siguiente colección de videos didácticos \url{https://www.youtube.com/playlist?list=PL8bSwVy8_IcPCsBE71CYBLbQSS8ckWm6x}}\\:

\begin{enumerate}
	\item \emph{Señal}: se trata de una función que describe un fenómeno físico, y que se emplea para la transmisión de información.

    \item \emph{Dominio temporal}: establece el eje de abcisas con el tiempo.

    \item \emph{Dominio de la señal}: determina si la señal se expresa en tiempo o en frecuencia (a través de transformadas).

    \item \ac{ADC}: elemento electrónico que permite la conversión de señales analógicas a señales digitales.

    \item \ac{RSSI}: permite establecer el nivel de potencia de una señal, con respecto a 1 mW de potencia. Se expresa en dBm.

    \item \ac{SNR}: métrica que permite medir la potencia de la señal con respecto al ruido ambiente.

    \item \emph{Frecuencia}: parámetro de la función que define a la señal, el cual determina el número de veces que se repite en un segundo. Se mide en hercios (Hz).

    \item \emph{TX}: Se refiere a la transmisión de la señal.

    \item \emph{RX}: Hace referencia a la recepción de la señal.
\end{enumerate}

A continuación se muestran las referencias a las figuras de este trabajo junto con la fuente de la que han sido obtenidas:

%|m{3cm}|p{9cm}|

\begin{table}[H]
\begin{center}
\begin{tabular}{|p{0.3\textwidth}|p{0.7\textwidth}|}
\hline
\textbf{Referencia imágenes} & \textbf{Fuente de la que se ha obtenido}\\

\hline

\ref{fig:robots_autonomos} & \textbf{1.}\url{https://revistaderobots.com/robots-y-robotica/que-es-la-robotica/?cn-reloaded=1} \\

& \textbf{2.}\url{https://www.elindependiente.com/vida-sana/2018/01/22/los-robots-que-nos-cuidaran-en-2050/} \\

& \textbf{3.}\url{https://www.iguanarobot.com/wp-content/uploads/2021/03/429190-1.jpg} \\

& \textbf{4.}\url{https://www.robotexplorador.com/} \\

& \textbf{5.}\url{https://www.edsrobotics.com/blog/robots-autonomos-que-son/} \\

\hline

\ref{fig:campos_robotica} & \textbf{1.}\url{https://www.hogarmania.com/hogar/economia/como-elegir-mejor-robot-aspirador.html} \\

& \textbf{2.}\url{http://automata.cps.unizar.es/robotica/Morfologia.pdf} \\

& \textbf{3.}\url{https://www.aarp.org/espanol/salud/enfermedades-y-tratamientos/info-12-2013/cirugia-robotica-beneficios-riesgos.html} \\

& \textbf{4.}\url{https://www.nobbot.com/mars-home-planet-reto-mundial-colonizar-marte/} \\

\hline

\ref{fig:robotica-tacto-bolonia} & \url{https://elpais.com/eps/2023-05-27/robots-que-sienten-lo-que-tocan.html} \\

\hline

\ref{fig:AKROD} & \url{http://www.technovelgy.com/ct/Science-Fiction-News.asp?NewsNum=455} \\

\hline

\ref{fig:lokomat} & \url{https://www.medicalexpo.es/prod/hocoma/product-68750-438408.html} \\

\hline

\ref{fig:exoesqueleto_atlas} & \href{https://www.agenciasinc.es/Noticias/El-exoesqueleto-pediatrico-del-CSIC-ya-puede-ser-comercializado-internacionalmente}{\textbf{https://exoesqueleto/pediatrico/puede/comercializar}} \\

\hline

\end{tabular}
\end{center}
\end{table}

\begin{table}[H]
\begin{center}
\begin{tabular}{|p{0.3\textwidth}|p{0.7\textwidth}|}

\hline

\ref{fig:aplicaciones-exoesqueletos} & \textbf{1.}\url{https://altertecnia.com/exoesqueletos-mejorar-productividad/} \\

& \textbf{2.}\url{https://www.eafit.edu.co/innovacion/spinoff/natural-vitro/PublishingImages/banner\%20-exoesqueleto.jpg} \\

& \textbf{3.}\url{https://www.marsibionics.com/atlas-pediatric-exo-pacientes/} \\

& \textbf{4.}\href{https://www.elespanol.com/invertia/disruptores-innovadores/disruptores/startups/20230526/primer-exoesqueleto-militar-spain-soldado-lesiones-batalla/766173641_0.html}{\textbf{https://exoesqueleto-militar}} \\

\hline

\ref{fig:exos-flexibles-rigidos} & \textbf{1.}\url{https://shop.bihar.coop/es/inicio/34-exoesqueleto-flexible.html} \\

& \textbf{2.}\href{https://www.renishaw.es/es/project-march-construye-exoesqueletos-que-facilitan-la-movilidad-de-personas-con-lesiones-medulares--46996}{\textbf{https://exoesqueleto/rigido/movilidad}} \\

\hline

\ref{fig:puntos-clave-imagen} & \href{https://mlhive.com/2021/11/person-pose-landmarks-detection-using-mediapipe}{\textbf{https://person-pose-keypoints}} \\

\hline

\ref{fig:logo_stm32cubeide} &  \url{https://www.st.com/content/st_com/en/stm32cubeide.html} \\

\hline

\ref{fig:exos_activo_pasivo} & \textbf{1.}\url{https://journals.sagepub.com/doi/full/10.1177/1687814017735791} \\

& \textbf{2.}\url{https://cjme.springeropen.com/articles/10.1186/s10033-020-00465-z} \\

\hline

\ref{fig:esquema_redes_neuronales} & \url{https://www.atriainnovation.com/que-son-las-redes-neuronales-y-sus-funciones/} \\

\hline

\ref{fig:esquema_arbol_decision} & \href{https://aprendeia.com/arboles-de-decision-regresion-teoria-machine-learning/}{\textbf{https://esquema/arbol/decision}} \\

\hline

\ref{fig:obtencion_fases_keypoints_movenet} & \href{https://medium.com/axinc-ai/movenet-pose-estimation-for-video-with-intense-motion-2b92f53f3c8}{\textbf{https://movenet/pose/estimation}} \\

\hline

\ref{fig:esquema_keypoints_mediapipe} & \url{https://www.analyticsvidhya.com/blog/2022/03/pose-detection-in-image-usi} \\

\hline

\ref{fig:ejemplos_vision_mediapipe} & \url{https://developers.google.com/mediapipe/solutions/examples} \\

\hline

\end{tabular}
\caption{Anexo con las fuentes de donde se han obtenido las imágenes para este proyecto}
\label{cuadro:anexo_imagenes_fuentes}
\end{center}
\end{table}


