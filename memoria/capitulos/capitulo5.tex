\chapter{Conclusiones}
\label{cap:capitulo5}

En esta última sección, se comentarán las ideas extraidas tras el desarrollo del proyecto, así como el conocimiento obtenido y la posible continuación del mismo.

\section{Objetivos cumplidos}
\label{sec:objetivos_cumplidos}

Entre los objetivos planteados y resueltos encontramos los siguientes:

\begin{enumerate}
    \item Desarrollo de una solución \ac{ROS} para una aeronave no tripulada o dron.
    
    \item Creación de una aplicación reactiva de simulación de señales, siguiendo el modelo de Friis, y empleando Matplotlib.
    
    \item Extracción del mejor algoritmo (entre los probados), a través de métricas comparativas y experimentación, haciendo uso de Gazebo 11, rviz y técnicas de aprendizaje por refuerzo (Q-Learning).
\end{enumerate}

\section{Balance global y competencias adquiridas}
\label{sec:balance_global_competencias_adquiridas}

En cuanto a los conocimientos adquiridos, podemos distinguir:

\begin{enumerate}
    \item Desarollo de aplicaciones usando OpenCV y Matplotlib.
    
    \item Uso de plugins y trabajo con el modelo SDF del Iris Drone.
    
    \item Creación de entornos personalizados empleando Gazebo 11.
    
    \item Uso de marcadores y del módulo grid\_map para rviz, así como su aplicación conjunta a través ROS en C++ y Python.
    
    \item Empleo de PX4 y MAVROS para el control de la aeronave.
    
    \item Desarrollo de soluciones basadas en Q-Learning a través de Python.
    
    \item Adquisición de conocimientos relacionados al estudio de señales y su comportamiento.
\end{enumerate}

\section{Líneas futuras}
\label{sec:lineas_futuras}

Finalmente, y tal y como se comentó en la sección anterior, la manera de continuar este proyecto es haciéndolo más afín a entornos realistas, lo que implica agregar elementos como obstáculos y perturbaciones, así como añadir modos de funcionamento para el dron que le permita mejorar su adaptabilidad.\\

Además, el mejor algoritmo según los resultados obtenidos alude a la solución por Q-Learning, sin embargo, se podría considerar emplear otras variantes (como empleando aprendizaje profundo) y comprobar si arrojan mejores resultados.