\chapter{Conclusiones}
\label{cap:capitulo5}

En esta última sección, se comentarán las ideas extraidas tras el desarrollo del proyecto, así como el conocimiento obtenido y la posible continuación del mismo.

\section{Objetivos cumplidos}
\label{sec:objetivos_cumplidos}

Durante el desarrollo de este \ac{TFG}, se ha conseguido satisfactoriamente desarrollar una solución de navegación autónoma para comandar a un dron hacia un transmisor de \ac{RF}, demostrando que la aproximación realizada con \ac{RL} cumple con lo esperado. En detalle se puede afirmar que:

\begin{enumerate}
    \item Se ha desarrollado una interfaz capaz de interactuar de manera reactiva con el dron, comandando tanto posiciones como velocidades.
    \item Se ha creado un nodo \ac{ROS} disponible para la comunidad\footnote[1]{\url{https://github.com/RoboticsLabURJC/2022-tfg-cristian-sanchez/blob/main/src/heatmap_util/scripts/rf_data_server.py}}, el cual gestiona un modelo de propagación de señales, a demanda de las aplicaciones que lo usen.
    \item Se ha conseguido implementar diversos algoritmos capaces de navegar hacia transmisores \ac{RF} de manera autónoma.
    \item Se ha demostrado que el algoritmo basado en Q-Learning es el más eficiente en términos de tiempo, número de movimientos, y movimientos malos (hacia potencias de señal inferiores), de todos los planteados.
    \item Se ha adaptado la mejor solución a un entorno realista donde se presentan obstáculos, a través de un enfoque híbrido empleando \ac{VFF}.
\end{enumerate}

\section{Balance global y competencias adquiridas}
\label{sec:balance_global_competencias_adquiridas}

En cuanto a los conocimientos adquiridos, podemos distinguir:

\begin{enumerate}
    \item Desarollo de aplicaciones usando OpenCV y Matplotlib.    
    \item Uso de plugins y trabajo con el modelo SDF del Iris Drone.    
    \item Creación de entornos personalizados empleando Gazebo 11.    
    \item Uso de marcadores y del módulo grid\_map para rviz, así como su aplicación conjunta a través ROS en C++ y Python.
    \item Empleo de PX4 y MAVROS para el control de la aeronave.    
    \item Desarrollo de soluciones basadas en Q-Learning a través de Python.    
    \item Adquisición de conocimientos relacionados al estudio de señales y su comportamiento.
\end{enumerate}

\section{Líneas futuras}
\label{sec:lineas_futuras}

Finalmente, y tras demostrar que el mejor algoritmo según los resultados obtenidos alude a la solución por Q-Learning, cabe preguntarse si se podrían encontrar variantes relacionadas (siguiendo la línea de \ac{RL}) que arrojasen mejores resultados, es decir, ver si rinden mejor que las soluciones planteadas, vease \ac{PPO}, \ac{DDPG}, \ac{SAC}, entre otros. Todo ello, con el fin último de transladarlo a un dispositivo real y verificar su funcionamiento.